\section{Sources of uncertainties}
\label{sec:sources-of-uncertainties}
Uncertainties are an inherent part of any measurement or calculation, and it is important to properly account for them to obtain reliable results.There are several sources of uncertainties that need to be considered. These include statistical uncertainties, systematic uncertainties, and model uncertainties. Statistical uncertainties arise from the limited precision of our measurements or the finite size of our data samples. Systematic uncertainties, on the other hand, arise from imprecision in our experimental setup or analysis techniques. Model uncertainties arise from the limitations of our theoretical models used.

% Individual sources of uncertainties are considered to be uncorrelated.
% Uncertainties are considered correlated between different regions.
% Gaussian constraints are used for the systematic uncertainties, and Poisson constraints are used for the statistical uncertainties.
% They are included in the fit as nuisance parameters, which are constrained to be within their uncertainties.
% The nuisance parameters are profiled in the fit, and the uncertainties are propagated to the final results.
%
