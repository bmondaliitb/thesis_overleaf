
\section{Categorisation of photons}
\label{sec:photon-categorisation}

The main interest of this analysis are \ttbar events where an additional photon is generated in the hard-scattering event, also called a \emph{prompt} photon.
However, photons can occur at many other stages of what is being recorded as an \enquote{event} with the ATLAS detector.
In addition, other particles and activities in the detector may fake photon signatures and be identified as such.
Among the photon candidates detected and reconstructed with the ATLAS detector, this analysis distinguishes three classes:
%
\begin{enumerate}
\item Prompt photons originating from the hard-scattering event.
\item Electron-fake photons, in the following denoted as \emph{\efakes}, which are electrons faking a photon signature in the calorimeter.
\item Non-prompt photons originating from hadrons, for example $\pi^0 \to \gamma\gamma$ decays, and photon signatures faked by hadronic energy depositions in the calorimeter are considered in one category, referred to as hadron-fake photons, in the following denoted as \emph{\hfakes}.
\end{enumerate}
%
To assess and estimate contributions to these three classes in simulation, \emph{MCTruthClassifier} information is used to identify the origin of a reconstructed photon candidate.
For this, the \texttt{xAOD::TruthHelpers} class is used to retrieve the truth object $\gamma_{\mathrm{truth}}^{\mathrm{cand}}$ associated to the photon candidate.
Then, according to the MCTruthClassifier::origin and MCTruthClassifier::truth values of that truth object, the photon candidate is classified into one of the three categories above.
To classify a candidate as \efake, \emph{any} of the following criteria must be fulfilled:
%
\begin{itemize}
\item PDG ID $= \pm 11$ (electron)
\item $\Delta R (\gamma_{\mathrm{truth}}^{\mathrm{cand}}, e_{\mathrm{truth}}) < 0.1$, where $e$ denotes a truth electron.
  Similar to \cref{sec:sample-overlap-removal}, a list of truth electrons is compiled and $\gamma_{\mathrm{truth}}^{\mathrm{cand}}$ must not overlap with any of them.
The truth-electron criteria are:
  \begin{enumerate}
  \item PDG ID $= \pm 11$
  \item $\pT > \SI{10}{\GeV}$
  \item $|\eta| < 3.0$
  \item barcode $< \num{100000}$
  \end{enumerate}
\end{itemize}
%
To avoid any possible double-classification, all photon candidates that match either of the two above \efake criteria, are categorised as such and \emph{not examined further}.
For all remaining candidates, if $\gamma_{\mathrm{truth}}^{\mathrm{cand}}$ meets any of the following criteria, the candidate is classified as \hfake:
%
\begin{itemize}
\item MCTruthClassifier::type $= 16$ and MCTruthClassifier::origin $\in [23, 35]$.
These are photons of type \emph{background photon} originating from baryons or mesons.
\item MCTruthClassifier::type $= 16$ and MCTruthClassifier::origin $= 42$. These are photons of type \emph{background photon} originating from $\pi^0 \to \gamma\gamma$ decays.
\item MCTruthClassifier::type $= 17$, corresponding to hadronic energy deposition.
\end{itemize}
%
If the photon candidate classifies as neither \efake nor \hfake, it is treated as prompt.


\subsection{Grouping of processes}
%\label{sec:process-grouping}

For the convenience of the analysis the signal and background processes are grouped according to the needs as shown in the Table \ref{tab:process-grouping}. 

\begin{table}[hp]
  \centering
  \caption{The grouping of processes}
  \label{tab:process-grouping}
  \scriptsize
  \begin{tabular}{ll}
    \toprule
    Groups & Processes \\
    \midrule
     & Prompt $\gamma$ from\\
    \tty prod (sig) & \tty production\\
    \tty decay & \tty decay \\
    $Wt\gamma$& $Wt\gamma$\\
    & $Wt$ (non-overlapped region)\\
    Prompt $\gamma$ & Di-boson(+$\gamma$) \\
    & \ttbar(+$\gamma$) (non-overlapped region) \\
    & $t\bar{t}W$(+$\gamma$), $t\bar{t}Z$(+$\gamma$) \\
    & Z$\gamma$ \\
    & W$\gamma$ \\
    & Z+jets(+$\gamma$) (non-overlap region) \\
    & W+jets(+$\gamma$) (non-overlap region) \\
    & single top (+$\gamma$) [s,t] \\
    & \\
    \hfake $\gamma$ & Hadron fake $\gamma$ from all processes\\
    & \\
    \efake $\gamma$ & Electron fake $\gamma$ from all processes\\
    Lepton fake & Events with fake leptons from all processes\\
    \bottomrule
  \end{tabular}
\end{table}
\FloatBarrier


