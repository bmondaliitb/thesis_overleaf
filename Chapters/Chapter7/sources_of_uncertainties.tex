\chapter{Sources of systematic uncertainties}
\label{sec:sources-of-uncertainties}
Uncertainties are an inherent part of any measurement, and it is important to properly account for them. There are several sources of uncertainties that need to be considered. These include statistical uncertainties, systematic uncertainties, and model uncertainties. Statistical uncertainties arise from the finite size of our data samples. In collider experiments, events occur randomly with a constant probability. The number of events in a given sample is a random variable that follows a Poisson distribution. The statistical uncertainty is the standard deviation of the Poisson distribution. This is inherent to the data and can only be reduced by increasing the size of the data sample meaning more integrated luminosity.

Systematic uncertainties, on the other hand, arise from imprecision in our experimental setup or analysis techniques. For example, the uncertainty can come from the particle reconstruction and identification, their energy and momentum measurement, luminosity measurement and so on.

Model uncertainties arise from the limitations of our theoretical models used to describe the data. MC event generation is used to generate event from a theoretical model. The uncertainty arises from the choice of the PDF set, the choice of the renormalization and factorization scales, the choice of the parton showering and hadronization models, and so on. Also, the cross-section of every process has an uncertainty associated with it. Different sources of uncertainties are discussed in the following sections. 

\paragraph{Treatment of uncertainties} As mentioned earlier the histogram templates obtained from MC simulated events are used for modelling the signal and background processes. MC templates are produced for different sources of uncertainties. The difference between the nominal template and systematic variation template taken as a measure of the uncertainty. % the question arises how we create the systematic variation templates?
In the statistical model (likelihood) different sources of uncertainties can be assumed as separate auxiliary measurements and can be incorporated in the likelihood function as nuisance parameters (NP), which are constrained to be within their uncertainties. For every source of uncertainty, one NP is added to the fit model. Uncertainties arising from different sources are treated as independent and uncorrelated. This means that the impact of one source of uncertainty does not affect the impact of another source. However, uncertainties within the same source are assumed to be correlated across different signal and control regions. Systematic uncertainties are modeled using Gaussian constraints. On the other hand, statistical uncertainties are modeled using Poisson constraints. The uncertainties associated with the nuisance parameters are estimated from the fit and propagated to the final results.
%discuss about morphing

\paragraph{Smoothing}The smoothing technique is used for some sources of uncertainties to reduce the impact of statistical fluctuations in the template.

\paragraph{Symmetrization}The are some uncertainties for which there are up and down variations w.r.t nominal is available and for some uncertainties only one variation is available. For asymmetric up and down variation, the variation is symmetrized to have a simpler implementation in the fit model. Uncertainties with up and down variations are symmetrized using the Two-Sided symmetrization method in the following way $$ \mathrm{symmetrized \ up/down} = \mathrm{nominal} \pm (\mathrm{up - down})/2 $$ In cases where only one variation is available, the other variation is obtained by mirroring the available one around the nominal. 

\paragraph{Pruning}Uncertainties with negligible impact are pruned from the fit model to avoid using too many NPs in the fit model. The pruning decision is made for shape and normalization components separately. The uncertainty can have a negligible impact on the normalization but a significant impact on the shape. In such cases, the uncertainty is pruned from the normalization component but kept in the shape component. It applies to the other way around as well. For pruning based on the normalization the integral of the template is considered and the relative difference between the nominal is calculated, if the relative difference is less than 0.1\% the uncertainty is pruned. For pruning based on the shape, the relative difference w.r.t nominal is calculated for every bin, if for any bin the difference is more than 0.1\% the uncertainty is kept.


\section{Experimental uncertainties}
\label{sec:experimental_uncertainties}
Experimental uncertainties arise from many sources these include uncertainties on the reconstruction and identification of physics objects, uncertainties on the energy and momentum scales and resolutions, uncertainties on the flavour tagging of jets, uncertainties on the integrated luminosity measurement, and uncertainties on the pile-up simulation. These affect both the signal and background simulations. 

\paragraph{Lepton efficiencies}
Electrons are reconstructed from the energy deposits in the calorimeter and the tracks in the inner detector. Muons are reconstructed combining the independent measurement in inner detector and muon spectrometer. Experimentally determined energy spectra are corrected with the selection efficiency, these include efficiency related to trigger, particle isolation, identification and reconstruction. The accuracy with which the detector simulation models the observed electron efficiencies is very important. Slight mismodelling is observed in the efficiencies between MC and in data. Using tag-and-probe approach and using \zee($\mu\mu$ for muons)  or \jpsiee($\mu\mu$ for muons) processes~\cite{ATLAS:2019jvq,ATLAS:2016lqx} the efficiencies are further corrected. Multiplicative scale factors are obtained from the ratio of efficiencies in data and MC. Scale factors are obtained in the bins of kinematic observables. The scale factors are applied to the MC to correct for the difference in efficiencies. The uncertainty on the estimated scale factors are used to create systematic variation templates.

\paragraph{Photon efficiencies}
Photons are reconstructed from the energy deposits in the calorimeter. The identification of prompt photon in the hadronic environment is challenging, because of the presence of overwhelming amount of  non-prompt photon from hadron decays. Prompt photons are identified with selection criteria on the shape and properties of the electromagnetic showers and also by requiring them to be isolated from other particles. These identification and isolation efficiencies in MC are further corrected using scale factors estimated using data-driven approach ~\cite{ATLAS:2016ecu,ATLAS:2019qmc}. The scale factors are varied up and down by their uncertainties to create systematic variation templates. 

\paragraph{$e,\mu,\gamma$ energy and momentum calibration}
Electrons and photon energy are measured from EM calorimeter clusters. The energy and momentum scale and resolution are calibrated using \zee decay and are validated using radiative $Z-$boson decays(details in~\cite{ATLAS:2019qmc}). The muon momentum is studied using \zmumu and \jpsimumu processes, and correction factors are derived to correct the muon momentum scale and resolution in MC to match with the data~\cite{ATLAS:2016lqx}. The uncertainties on the calibration are used to create systematic variation templates to propagate the uncertainties to the analysis.


\subsection*{Jet uncertainties}
% jets are reconstructed using the topo-cluster formed from the energy deposits in the calorimeter as well as from the tracks in the inner detector, combining these two information is referred to as ATLAS particle-flow algorithm.
% jets are first calibrated with sequence of simulation-based corrections then several in situ techniques are employed to correct the differences between data and simulation.
Jet reconstruction from the detector signature of the stable hadrons is described in~\cref{sec:physics_object_reconstruction}. The uncertainties comes mainly from the jet energy scale and resolution, jet vertex tagging. Jet energy scale which calibrates the jet energy with particle level energy of the jet. For this jets are calibrated first with sequence of simulation-based corrections then several \emph{in-situ} techniques are used to correct the differences between data and simulation. The individual steps correct various effects, such as, origin correction which corrects four momentum to point to primary vertex instead of pointing to the center of the detector, pile-up correction which removes the excess energy due to in-time and out-of-time pileup, absolute MC-based calibration which corrects jet 4-momentum to the particle-level energy scale, global sequential calibration which reduces flavor dependence and energy leakage effects using calorimeter, track and other variables. Finally a residual calibration is derived using \emph{in-situ} measurements and is applied only to data. The uncertainties may have many sources and are reduced to fewer effective nuisance parameter through eigenvector decomposition. Its uncertainty is split into several independent categories: modelling and statistical uncertainties on the extrapolation of the jet calibration from the central region, jet flavour composition, high-\pT jet behaviour, \bjet energy scale uncertainties, uncertainties due to pile-up, uncertainties on \textit{in situ} jet energy corrections, etc. In one category, there are usually more than one physical source of the uncertainty. To study the JES uncertainty, each source is varied up and down independently by its corresponding uncertainty. 

The jet energy resolution (JER) is measured using the balance between jets and well measured objects like photons or $Z$ bosons, and it is found to be in agreement between data and MC. There are a total of seven effective nuisance parameters associated to JER in the category reduction scheme, and a single source of uncertainty for the agreement between data and Monte Carlo, all of which are varied by one sigma to study their impact on the analysis.

The systematic uncertainty associated to the jet vertex tagging (JVT) is obtained by varying up and down the JVT cut using the \emph{JetVertexTaggerTool}~\cite{ATLAS-CONF-2014-018}. 


\subsection*{\btag uncertainties}

Jets coming from a \bquarks have their own topological features, and \btag allows to distinguish them from light-flavour jets. With \emph{\btag}, each jet can be assigned to a different working point, and the \btag uncertainties on this jet are derived for this specific working point. They are accounted for by varying the calibration scale factors for \ensuremath{b}-,\ensuremath{c}-, and light-flavour jets up and down by their corresponding systematic uncertainties independently. For each jet category, the uncertainties are decomposed into several uncorrelated components using the eigenvector method. As pseudo-continuous \btag weights are used the corresponding eigenvectors are used, which results in 45 nuisance parameters for \bjets and 20 nuisance parameters for \cjets and light-flavour jets each~\cite{ATL-PHYS-PUB-2017-013}.
%For example, there are 9, 4 and 4 eigenvectors for \bjets, \cjets and light-flavour jets uncertainties, respectively~\cite{ATL-PHYS-PUB-2017-013}.
As the pseudo-continuous b-tagging is only calibrated for b-jets below 400 GeV, c-jets below 250 GeV and light jets below 300 GeV, a normalisation uncertainty was assigned to events containing at least one uncalibrated jet. For each type of jets one uncertainty was defined and for each event containing an uncalibrated jets of that type (b-jet, c-jets or light jet) a 50\% uncertainty was assigned. This leads to three systematic uncertainties.

\subsection*{Missing transverse momentum}

The \met is reconstructed~\cite{ATLAS-CONF-2018-023} from the vector sum of several terms corresponding to different types of reconstructed objects. The estimated uncertainties for electrons, muons, photons and jets are propagated into the uncertainty of \met. Thus, the only new contribution is the systematic uncertainty of the soft terms $E_{\text{x,y}}^{\text{RefSoftJet}}$ and $E_{\text{x,y}}^{\text{CellOut}}$.

The systematic uncertainty of the soft-term scale is estimated by comparing the ratio of Monte-Carlo simulation to data. The average deviation of the ratio from unity is taken as a flat uncertainty on the absolute scale. The systematic uncertainty of the soft-term resolution is estimated by evaluating the level of agreement between data and MC in the $E^{\text{miss}}_{\text{x}}$ and $E^{\text{miss}}_{\text{y}}$ resolution. Both the scale and resolution of the soft term are varied up and down by one standard deviation to study their impact on the analysis. 

\subsection*{Pile-up uncertainties}

The uncertainty on the reweighting procedure used to correct the pile-up profile in MC to match the data, is based on the disagreement between the instantaneous luminosity in data~\cite{DAPR-2013-01} and in simulation. Both the nominal and systematically-shifted pileup-reweighting weights are obtained using the standard \texttt{PileupReweighting} tool~\footnote{More details: \url{https://twiki.cern.ch/twiki/bin/viewauth/AtlasProtected/ExtendedPileupReweighting}.}. 

\subsection*{Luminosity uncertainties}

As quoted in \cref{sec:selection}, the total integrated luminosity has an uncertainty of \intlumiunc.



\subsection{Model uncertainties}
\label{sec:theoretical_uncertainties}



