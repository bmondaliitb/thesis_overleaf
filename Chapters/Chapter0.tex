% Chapter Template

\chapter{Introduction} % Main chapter title

\label{Chapter0} % Change X to a consecutive number; for referencing this chapter elsewhere, use \ref{ChapterX}

This thesis presents a measurement of the differential cross-sections of the \tty process. In this section, I present a complete overview of the thesis. The theory of Standard Model (SM) explains the physics at a very small length scale. In this model, the universe is made of fundamental particles quarks and leptons and these fundamental particles interact with each other through fundamental forces in nature, which are electromagnetic force, weak force, strong force and gravitational force(although SM doesn't explain gravitational force). Among these fundamental particles, the top quark is the heaviest particle. The top quark participates in electromagnetic interaction, strong interaction and weak interaction. The top quark has a very short lifetime (~$10^-{25}$s). The only way to produce a top quark is in a very high-energy collision of fundamental particles in a particle collider. The more on top quark theory is explained in Chapter ~\ref{Chapter1}.
This thesis aims to better understand top quark and photon interaction using the \tty process. Through precise measurement of the \tty cross-section, we can test the validity of the SM. Proton-proton collision data at a center-of-mass energy of 13 TeV collected by the ATLAS detector during its Run2 operation (2015-2018) has been used to measure the cross-section of this process. More on LHC and the ATLAS detector is presented in Chapter ~\ref{Chapter2}. In total ATLAS detector recorded 140.1 $fb^{-1}$ luminosity of data.


